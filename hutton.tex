\usetheme{default}
\usepackage{xcolor}
\usepackage{listings}

\usebackgroundtemplate{
\includegraphics[width=\paperwidth,height=\paperheight]{images/hutton_background}
}
%% PRESENTATION CONFIGURATION PARAMETERS %%%%%%%%%%%%%%%%%%%%%%%%%%%%%%%%%%%%%%%
%\titlebackgroundfile{images/hutton_title}
%\framebackgroundfile{images/hutton_background}

% 120 R, 162 G, 47 B
\definecolor{hutton_green}{HTML}{78A22F}
% 135 R, 33 G, 81 B
\definecolor{hutton_purple}{HTML}{872175}
% 86 R, 155 G, 190 B
\definecolor{hutton_blue}{HTML}{569BBE}

% 209 R, 222 G, 178 B
\definecolor{hutton_lightgreen}{HTML}{D1DEB2}
% 215 R, 174 G, 207 B
\definecolor{hutton_lightpurple}{HTML}{D7AECF}
% 195 R, 222 G, 234 B
\definecolor{hutton_lightblue}{HTML}{C3DEEA}

\hypersetup{colorlinks,linkcolor=blue,urlcolor=blue}

\usefonttheme{structurebold}
\setbeamercolor{alerted text}{fg=orange}
\setbeamercolor{background canvas}{bg=white}
\setbeamercolor{block title}{bg=hutton_purple}
\setbeamercolor{frametitle}{fg=hutton_purple}
\setbeamercolor{title}{fg=black}
\setbeamercolor{titlelike}{fg=hutton_green}
\setbeamercolor{author}{fg=hutton_purple}
\setbeamercolor{author in head/foot}{fg=white}
\setbeamercolor{title in head/foot}{fg=white}
\setbeamercolor{section in head/foot}{fg=hutton_purple}
\setbeamercolor{normal text}{fg=black}
\setbeamercolor{frametitle}{fg=hutton_purple}
\setbeamerfont{block title}{size={}}
\setbeamerfont{author}{size=\footnotesize}
\setbeamerfont{date}{size=\footnotesize}
\setbeamercolor{section in toc shaded}{fg=hutton_purple}
\setbeamercolor{section in toc}{fg=hutton_purple}
\setbeamercolor{subsection in toc shaded}{fg=hutton_purple}
\setbeamercolor{subsection in toc}{fg=hutton_purple}
\setbeamertemplate{itemize item}[circle]
\setbeamertemplate{itemize subitem}[circle]
\setbeamertemplate{itemize subsubitem}[circle]
\setbeamertemplate{itemize subsubsubitem}[circle]
\setbeamercolor{itemize item}{fg=hutton_purple}
\setbeamercolor{itemize subitem}{fg=hutton_purple}
\setbeamercolor{itemize subsubitem}{fg=hutton_purple}
\setbeamercolor{itemize subsubsubitem}{fg=hutton_purple}
\setbeamercolor{enumerate item}{fg=hutton_purple}
\setbeamercolor{enumerate subitem}{fg=hutton_purple}
\setbeamercolor{enumerate subsubitem}{fg=hutton_purple}
\setbeamercolor{enumerate subsubsubitem}{fg=hutton_purple}
\setbeamercolor{alerted text}{fg=hutton_green}
\setbeamerfont{alerted text}{series=\bfseries}
% This command makes sure that acrobat reader doesn't change the colours of the slide
% when there are figures with transparencies.
\pdfpageattr {/Group << /S /Transparency /I true /CS /DeviceRGB>>}

%Disables discrete bottom navigation bar
%\beamertemplatenavigationsymbolsempty

% Modify the slide titles to avoid the corner images,
\setbeamertemplate{frametitle}
{
\vspace{0.05\textheight}
\noindent\quad\begin{minipage}[t][0.12\textheight][t]{0.85\textwidth}
\insertframetitle\par
\end{minipage}
}

% Modify title page to avoid the big logo on right
\setbeamertemplate{title page}{
    \begin{picture}(0,0)
            %This ends up on top of the default background image, rather than replacing it:
            \put(-30,-165){%
                \includegraphics[width=\paperwidth,height=\paperheight]{images/hutton_title}
            }
            \put(0,-75){%
                \begin{minipage}[b][0.4\textheight][t]{0.75\textwidth}
                    \usebeamerfont{title}\usebeamercolor[fg]{title}{\inserttitle\par}
                    \usebeamerfont{subtitle}\usebeamercolor[fg]{subtitle}{\insertsubtitle\par}
                \end{minipage}
            }
            \put(0,-135){%
                \begin{minipage}[b][0.1\textheight][t]{\textwidth}
                    \usebeamerfont{author}\usebeamercolor[fg]{author}{\insertauthor\par}
                \end{minipage}
            }
            \put(0,-155){%
                \begin{minipage}[b][0.1\textheight][t]{\textwidth}
                    \usebeamerfont{institute}\usebeamercolor[fg]{institute}{\insertinstitute\par}
                \end{minipage}
            }
    \end{picture}
}

% Settings for code listings in lstlistings
\lstset{ %
  backgroundcolor=\color{hutton_lightgreen}, % choose the background color; you must add \usepackage{color} or \usepackage{xcolor}
  basicstyle=\ttfamily, % the size of the fonts that are used for the code
  breakatwhitespace=false, % sets if automatic breaks should only happen at whitespace
  breaklines=true, % sets automatic line breaking
  captionpos=b, % sets the caption-position to bottom
  commentstyle=\color{red}, % comment style
  deletekeywords={...}, % if you want to delete keywords from the given language
  escapeinside={\%*}{*)}, % if you want to add LaTeX within your code
  extendedchars=true, % lets you use non-ASCII characters; for 8-bits encodings only, does not work with UTF-8
  frame=single, % adds a frame around the code
  keepspaces=true, % keeps spaces in text, useful for keeping indentation of code (possibly needs columns=flexible)
  keywordstyle=\color{blue}, % keyword style
% language=Octave, % the language of the code
  morekeywords={*,...}, % if you want to add more keywords to the set
  numbers=none, % where to put the line-numbers; possible values are (none, left, right)
  numbersep=5pt, % how far the line-numbers are from the code
  numberstyle=\tiny\color{gray}, % the style that is used for the line-numbers
  rulecolor=\color{black}, % if not set, the frame-color may be changed on line-breaks within not-black text (e.g. comments (green here))
  showspaces=false, % show spaces everywhere adding particular underscores; it overrides 'showstringspaces'
  showstringspaces=false, % underline spaces within strings only
  showtabs=false, % show tabs within strings adding particular underscores
  stepnumber=1, % the step between two line-numbers. If it's 1, each line will be numbered
  stringstyle=\color{hutton_purple}, % string literal style
  tabsize=4, % sets default tabsize to 2 spaces
  title=\lstname, % show the filename of files included with \lstinputlisting; also try caption instead of title
  aboveskip=0pt, % tight vertical layout
  belowskip=0pt, % tight vertical layout
}
