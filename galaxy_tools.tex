\documentclass[table]{beamer}
\mode<presentation>
\usetheme{default}
\usepackage{multirow}

%Use James Hutton Style for Beamer
\usetheme{default}
\usepackage{xcolor}
\usepackage{listings}

\usebackgroundtemplate{
\includegraphics[width=\paperwidth,height=\paperheight]{images/hutton_background}
}
%% PRESENTATION CONFIGURATION PARAMETERS %%%%%%%%%%%%%%%%%%%%%%%%%%%%%%%%%%%%%%%
%\titlebackgroundfile{images/hutton_title}
%\framebackgroundfile{images/hutton_background}

% 120 R, 162 G, 47 B
\definecolor{hutton_green}{HTML}{78A22F}
% 135 R, 33 G, 81 B
\definecolor{hutton_purple}{HTML}{872175}
% 86 R, 155 G, 190 B
\definecolor{hutton_blue}{HTML}{569BBE}

% 209 R, 222 G, 178 B
\definecolor{hutton_lightgreen}{HTML}{D1DEB2}
% 215 R, 174 G, 207 B
\definecolor{hutton_lightpurple}{HTML}{D7AECF}
% 195 R, 222 G, 234 B
\definecolor{hutton_lightblue}{HTML}{C3DEEA}

\hypersetup{colorlinks,linkcolor=blue,urlcolor=blue}

\usefonttheme{structurebold}
\setbeamercolor{alerted text}{fg=orange}
\setbeamercolor{background canvas}{bg=white}
\setbeamercolor{block title}{bg=hutton_purple}
\setbeamercolor{frametitle}{fg=hutton_purple}
\setbeamercolor{title}{fg=black}
\setbeamercolor{titlelike}{fg=hutton_green}
\setbeamercolor{author}{fg=hutton_purple}
\setbeamercolor{author in head/foot}{fg=white}
\setbeamercolor{title in head/foot}{fg=white}
\setbeamercolor{section in head/foot}{fg=hutton_purple}
\setbeamercolor{normal text}{fg=black}
\setbeamercolor{frametitle}{fg=hutton_purple}
\setbeamerfont{block title}{size={}}
\setbeamerfont{author}{size=\footnotesize}
\setbeamerfont{date}{size=\footnotesize}
\setbeamercolor{section in toc shaded}{fg=hutton_purple}
\setbeamercolor{section in toc}{fg=hutton_purple}
\setbeamercolor{subsection in toc shaded}{fg=hutton_purple}
\setbeamercolor{subsection in toc}{fg=hutton_purple}
\setbeamertemplate{itemize item}[circle]
\setbeamertemplate{itemize subitem}[circle]
\setbeamertemplate{itemize subsubitem}[circle]
\setbeamertemplate{itemize subsubsubitem}[circle]
\setbeamercolor{itemize item}{fg=hutton_purple}
\setbeamercolor{itemize subitem}{fg=hutton_purple}
\setbeamercolor{itemize subsubitem}{fg=hutton_purple}
\setbeamercolor{itemize subsubsubitem}{fg=hutton_purple}
\setbeamercolor{enumerate item}{fg=hutton_purple}
\setbeamercolor{enumerate subitem}{fg=hutton_purple}
\setbeamercolor{enumerate subsubitem}{fg=hutton_purple}
\setbeamercolor{enumerate subsubsubitem}{fg=hutton_purple}
\setbeamercolor{alerted text}{fg=hutton_green}
\setbeamerfont{alerted text}{series=\bfseries}
% This command makes sure that acrobat reader doesn't change the colours of the slide
% when there are figures with transparencies.
\pdfpageattr {/Group << /S /Transparency /I true /CS /DeviceRGB>>}

%Disables discrete bottom navigation bar
%\beamertemplatenavigationsymbolsempty

% Modify the slide titles to avoid the corner images,
\setbeamertemplate{frametitle}
{
\vspace{0.05\textheight}
\noindent\quad\begin{minipage}[t][0.12\textheight][t]{0.85\textwidth}
\insertframetitle\par
\end{minipage}
}

% Modify title page to avoid the big logo on right
\setbeamertemplate{title page}{
    \begin{picture}(0,0)
            %This ends up on top of the default background image, rather than replacing it:
            \put(-30,-165){%
                \includegraphics[width=\paperwidth,height=\paperheight]{images/hutton_title}
            }
            \put(0,-75){%
                \begin{minipage}[b][0.4\textheight][t]{0.75\textwidth}
                    \usebeamerfont{title}\usebeamercolor[fg]{title}{\inserttitle\par}
                    \usebeamerfont{subtitle}\usebeamercolor[fg]{subtitle}{\insertsubtitle\par}
                \end{minipage}
            }
            \put(0,-135){%
                \begin{minipage}[b][0.1\textheight][t]{\textwidth}
                    \usebeamerfont{author}\usebeamercolor[fg]{author}{\insertauthor\par}
                \end{minipage}
            }
            \put(0,-155){%
                \begin{minipage}[b][0.1\textheight][t]{\textwidth}
                    \usebeamerfont{institute}\usebeamercolor[fg]{institute}{\insertinstitute\par}
                \end{minipage}
            }
    \end{picture}
}

% Settings for code listings in lstlistings
\lstset{ %
  backgroundcolor=\color{hutton_lightgreen}, % choose the background color; you must add \usepackage{color} or \usepackage{xcolor}
  basicstyle=\ttfamily, % the size of the fonts that are used for the code
  breakatwhitespace=false, % sets if automatic breaks should only happen at whitespace
  breaklines=true, % sets automatic line breaking
  captionpos=b, % sets the caption-position to bottom
  commentstyle=\color{red}, % comment style
  deletekeywords={...}, % if you want to delete keywords from the given language
  escapeinside={\%*}{*)}, % if you want to add LaTeX within your code
  extendedchars=true, % lets you use non-ASCII characters; for 8-bits encodings only, does not work with UTF-8
  frame=single, % adds a frame around the code
  keepspaces=true, % keeps spaces in text, useful for keeping indentation of code (possibly needs columns=flexible)
  keywordstyle=\color{blue}, % keyword style
% language=Octave, % the language of the code
  morekeywords={*,...}, % if you want to add more keywords to the set
  numbers=none, % where to put the line-numbers; possible values are (none, left, right)
  numbersep=5pt, % how far the line-numbers are from the code
  numberstyle=\tiny\color{gray}, % the style that is used for the line-numbers
  rulecolor=\color{black}, % if not set, the frame-color may be changed on line-breaks within not-black text (e.g. comments (green here))
  showspaces=false, % show spaces everywhere adding particular underscores; it overrides 'showstringspaces'
  showstringspaces=false, % underline spaces within strings only
  showtabs=false, % show tabs within strings adding particular underscores
  stepnumber=1, % the step between two line-numbers. If it's 1, each line will be numbered
  stringstyle=\color{hutton_purple}, % string literal style
  tabsize=4, % sets default tabsize to 2 spaces
  title=\lstname, % show the filename of files included with \lstinputlisting; also try caption instead of title
  aboveskip=0pt, % tight vertical layout
  belowskip=0pt, % tight vertical layout
}


\title[Writing Galaxy Tools]{Writing Galaxy Tools}
\author[Cock \textit{et al.}]{\underline{Peter Cock}$^1$,
Bj{\"o}rn Gr{\"u}ning$^2$,
Greg Von Kuster$^3$
}
\institute{$^1$ James Hutton Institute, Scotland, UK;
$^2$ Albert-Ludwigs-University, Germany;
$^3$ Penn State University, USA}
\date[30 June 2014]{30 June 2014, Galaxy Community Conference, Baltimore}
%Maybe I should tweak the theme more, but hack the subtitle for date:
\subtitle{30 June 2014, GCC2014, Baltimore}

% Conference:
%
% https://wiki.galaxyproject.org/Events/GCC2014
%
% Training Data: Writing Tools

\begin{document}

\frame[plain]{\titlepage}

\begin{frame}
\frametitle{Wrapping Command-line Tools}
\begin{itemize}
\item Tell Galaxy what options to show the user
\item Galaxy tells your tool the selected input filenames
\item Galaxy tells your tool the desired output filenames
\item Must tell Galaxy how to invoke the underlying tool...
\end{itemize}
\end{frame}

\begin{frame}
\frametitle{Heart of each Galaxy tool is an XML file}
Core elements:
\begin{itemize}
\item \texttt{<inputs>} -- parameters/options/files
\item \texttt{<outputs>} -- output files expected
\item \texttt{<command>} -- how to turn this into a command line string
\end{itemize}
Secondary elements:
\begin{itemize}
\item \texttt{<requirements>} -- tell Galaxy how to find the binaries etc
\item \texttt{<stdio>} -- what counts as an error?
\item \texttt{<description>} -- subtitle describing tool
\item \texttt{<help>} -- instructions to show the end user
\item \texttt{<tests>} -- functional tests
\end{itemize}
{\scriptsize \url{http://wiki.galaxyproject.org/Admin/Tools/ToolConfigSyntax}}
\end{frame}

\begin{frame}[fragile]
\frametitle{Heart of each Galaxy tool is an XML file}
Example:
\vspace{3mm}
{\scriptsize
\begin{lstlisting}[language=xml]
<tool id="my_tool" name="My Tool" version="0.0.1">
  <command>my_tool "$input1" "$output1"</command>
  <description>Run My Tool (patent pending)</description>
  <inputs>
    <param name="input1" type="data" format="fasta"
           label="Sequence in" help="FASTA format." />
  </inputs>
  <outputs>
    <data name="output1" format="fasta"
          label="My Tool Results" />
  </outputs>
  <help>
    This is a Galaxy wrapper for My Tool.
  </help>
</tool>
\end{lstlisting}
} %end size
\vspace{-3mm}
{\scriptsize \url{http://wiki.galaxyproject.org/Admin/Tools/ToolConfigSyntax}}
\end{frame}

\begin{frame}
\frametitle{The \texttt{<inputs>} and \texttt{<param ...>} tags}
\begin{itemize}
\item \texttt{<inputs>...</inputs>} contains \texttt{<param ...>} tag(s)
\item Each \texttt{<param ...>} tag requires a unique \texttt{name}
    \begin{itemize}
    \item This is used in the \texttt{<command>} and \texttt{<tests>}
    \end{itemize}
\item Each \texttt{<param ...>} tag requires a \texttt{type}
    \begin{itemize}
    \item e.g. for \texttt{type="data"} for an input file
    \end{itemize}
\item Each \texttt{<param ...>} tag should have a \texttt{label} and \texttt{help}
    \begin{itemize}
    \item These are shown in the user interface
    \end{itemize}
\item There are additional type-specific attributes
    \begin{itemize}
    \item e.g. for \texttt{type="data"} add \texttt{format="..."} for file type
    \end{itemize}
\end{itemize}
{\scriptsize \url{http://wiki.galaxyproject.org/Admin/Tools/ToolConfigSyntax}}
\end{frame}

\begin{frame}
\frametitle{The \texttt{<inputs>} and \texttt{<param ...>} tags}
Use \texttt{<param type="???" ...>} to control each parameter:
\begin{itemize}
\item \texttt{type="data"} -- input file (from current history)
%   \begin{itemize}
%   \item Additional \texttt{ftype="...."} attribute sets file types(s)
%   \end{itemize}
\item \texttt{type="text"} -- any string via a text box
\item \texttt{type="integer"} -- whole number via a text box
\item \texttt{type="float"} -- arbitrary number via a text box
\item \texttt{type="select"} -- Drop down lists, or radio buttons
%   \begin{itemize}
%   \item Contains \texttt{<option ...>} tags
%   \end{itemize}
\item \texttt{type="boolean"} -- True/false value with checkbox
\item \texttt{type="data\_column"} -- Pick column(s) from a tabular file
\item ...
%Am I missing any?
\end{itemize}
{\scriptsize \url{http://wiki.galaxyproject.org/Admin/Tools/ToolConfigSyntax}}
\end{frame}

\begin{frame}
\frametitle{The \texttt{<outputs>} and \texttt{<data ...>} tags}
\begin{itemize}
\item \texttt{<outputs>...</outputs>} contains \texttt{<data ...>} tag(s)
\item Each \texttt{<data ...>} tag requires a unique \texttt{name}
    \begin{itemize}
    \item This is used in the \texttt{<command>} and \texttt{<tests>}
    \end{itemize}
\item Each \texttt{<data ...>} tag should have an \texttt{ftype}
    \begin{itemize}
    \item e.g. for \texttt{ftype="fasta"} for a FASTA output file
    \end{itemize}
\item Each \texttt{<data ...>} tag should have a \texttt{label}
    \begin{itemize}
    \item This is the default dataset description in the history
    \end{itemize}
\end{itemize}
{\scriptsize \url{http://wiki.galaxyproject.org/Admin/Tools/ToolConfigSyntax}}
\end{frame}

\begin{frame}[fragile]
\frametitle{The \texttt{<command>} tag -- basics}
\begin{itemize}
\item The \texttt{<command>} tag is a command line string template
\item Named every input \texttt{<param ...>} and output \texttt{<data ...>}
\item Use \texttt{\$name} to refer to that input parameter or output file
    \begin{itemize}
    \item Ideally use \texttt{"\$name"} in case of spaces in filename
    \end{itemize}    
\item Can split \texttt{<command>} over multiple lines
\item XML so need \texttt{\&amp;}, \texttt{\&lt;} and \texttt{\&gt;}
\item Must escape \texttt{\$} to get an actual dollar sign, e.g.
\end{itemize}
\vspace{3mm}
\begin{lstlisting}[language=xml]
<command>
my_tool --threads \$GALAXY_SLOTS
"$input1" "$output1"
</command>
\end{lstlisting}
\vspace{-3mm}
{\scriptsize \url{http://wiki.galaxyproject.org/Admin/Tools/ToolConfigSyntax}}
\end{frame}

\begin{frame}[fragile]
\frametitle{The \texttt{<command>} tag -- advanced}
\begin{itemize}
\item The \texttt{<command>} tag uses the Cheetah template language
\item This can include for loops and if statements, e.g.
\end{itemize}
\vspace{3mm}
\begin{lstlisting}[language=xml]
<command>
my_tool --threads \$GALAXY_SLOTS
## double hash for comment lines
## single hash for Cheetah syntax
#if $output_choice=="long"
--long
#end if
"$input1" "$output1"
</command>
\end{lstlisting}
\vspace{-3mm}
%TODO - show Cheetah documentation link here
{\scriptsize \url{http://wiki.galaxyproject.org/Admin/Tools/ToolConfigSyntax}}
\end{frame}

\begin{frame}[fragile]
\frametitle{Advanced parameter options}
\begin{itemize}
\item Galaxy supports conditional and repeated constructs
    \begin{itemize}
    \item Defined with more XML in the \texttt{<inputs>} section
    \end{itemize}
\item Galaxy supports multiple input files as one parameter
\item Requires Cheetah syntax in the \texttt{<command>} tag
\item Easiest to learn by example?
\end{itemize}
\vspace{3mm}
\begin{lstlisting}
$ cd galaxy-dist
$ grep "<conditional " tools/*/*.xml
...
$ grep "<repeat " tools/*/*.xml
...
\end{lstlisting}
\vspace{-3mm}
{\scriptsize \url{http://wiki.galaxyproject.org/Admin/Tools/ToolConfigSyntax}}
\end{frame}

\begin{frame}[fragile]
\frametitle{The \texttt{<help>} tag}
\begin{itemize}
\item Uses reStructuredText markup language
\item Blank line for a paragraph break
\item Use asterisks for \texttt{*italics*}, double-asterisks for \texttt{**bold**}
\item Can include tables, images, links, etc.
%\item Inside XML so need \texttt{\&amp;}, \texttt{\&lt;} and \texttt{\&gt;}
\end{itemize}
\vspace{3mm}
{\scriptsize
\begin{lstlisting}[language=xml]
<tool id="my_tool" name="My Tool" version="0.0.1">
  ...
  <help>
    This is a Galaxy wrapper for *My Tool*.
    
    If you use this tool, **please cite this paper**:
    ...
  </help>
</tool>
\end{lstlisting}
} %end size
\vspace{-3mm}
%TODO - show RST documentation link here
{\scriptsize \url{http://wiki.galaxyproject.org/Admin/Tools/ToolConfigSyntax}}
\end{frame}

\begin{frame}[fragile]
\frametitle{The \texttt{<stdio>} tag}
\begin{itemize}
\item The \texttt{<stdio>} tag controls error detection
\item Galaxy default is any output on \textbf{stderr} means an error (!)
\item Unix/Linux convention allows logging etc on \textbf{stderr}
\item Unix/Linux convention is non-zero return code means error
\end{itemize}
For Unix style tools, only check the return code:
\vspace{3mm}
{\scriptsize
\begin{lstlisting}[language=xml]
<tool id="my_tool" name="My Tool" version="0.0.1">
  ...
  <stdio>
    <!-- Anything other than zero is an error -->
    <exit_code range="1:" />
    <exit_code range=":-1" />
  </stdio>
  ...
</tool>
\end{lstlisting}
} %end size
\vspace{-5mm}
{\scriptsize \url{http://wiki.galaxyproject.org/Admin/Tools/ToolConfigSyntax}}
\end{frame}

\begin{frame}
\frametitle{Further reading}
\begin{itemize}
\item Functional Tests
\item Dynamics captions on output files
\item Variable numbers of output files
\item Composite datatypes
\item Defining new Galaxy datatypes
\item Dependencies \& The Tool Shed
\item Galaxy macros to reduce repetitive XML
\end{itemize}
\end{frame}

\end{document}
